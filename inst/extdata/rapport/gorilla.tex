\subsection*{Utilisation de GOrilla}

\begin{description}
	\item [Étape 1 :]  choix de l'organisme parmis
		\begin{itemize}
			\item \emph{Homo sapiens}
			\item \emph{Mus musculus}
			\item \emph{Rattus norvegicus}
			\item \emph{Arabidopsis thaliana}
			\item \emph{Saccharomyces cerevisae}
			\item \emph{Caenorhabditis elegans}
			\item \emph{Drosophila melanogaster}
			\item \emph{Danio reiro}
		\end{itemize}
	\item [Étape 2 :] Choisir le mode \emph{Two unranked lists of genes (target and background lists)}
	\item [Étape 3 :] Fichiers d'entrée  
		\begin{description}
			\item [Target set : ] Liste de gènes, typiquement la liste d'un cluster.
			\item [Background set : ] Liste de toutes les gènes de la puce.
		\end{description}
	\item [Étape 4 Ontology : ] Sélectionner \emph{All}
	\item [Advanced parameters : ]  Cocher 
		\begin{itemize}
			\item \emph{Output results in Microsoft Exel format}
			\item \emph{Show also in REVIGO}
		\end{itemize}
	\item[Executer GOrilla] \emph{Search Enriched GO terms}
\end{description}
